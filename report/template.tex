\documentclass[12pt]{article}

% Packages
\usepackage[margin=1in]{geometry}
\usepackage{amsmath, amssymb}
\usepackage{graphicx}
\usepackage{float}
\usepackage{hyperref}
\usepackage{enumitem}
\usepackage{titlesec}

% Formatting
\titleformat{\section}{\normalfont\Large\bfseries}{\thesection.}{1em}{}
\titleformat{\subsection}{\normalfont\large\bfseries}{\thesubsection.}{1em}{}

% Title Info
\title{\textbf{ECE 602: Introduction to Optimization} \\ Final Project Report}
%\author{Team Name: \\ Team Members: \\ Student IDs:}
\date{}

\begin{document}

\maketitle

\section*{Project Overview}
Your final project consists of two interconnected parts aimed at deepening your understanding of numerical optimization and its application in solving real-world engineering problems:

\subsection*{Part 1: Reproducible Research}
\begin{itemize}
  \item Each project team has been assigned a peer-reviewed journal publication.
  \item Your task is to carefully study the paper, understand the optimization method used, and \textbf{reproduce the key results}.
  \item You may use either real or simulated data (as in the original paper or a close approximation).
  \item Compare your results to those in the paper and \textbf{comment on any discrepancies}, difficulties encountered, or assumptions made.
  \item Reflect critically on the method: What are its strengths and limitations? Is it robust to parameter changes or noise? Would you trust this method in a real-world scenario?
\end{itemize}

\subsection*{Part 2: Code Recycling \& Extension}
\begin{itemize}
  \item Use the method (or a variant of it) from Part 1 to solve a \textbf{different engineering or applied science problem}.
  \item You may draw on existing datasets, simulation tools, AI models, or recent literature to define your new problem.
  \item Creativity and originality are encouraged. The aim is to produce novel insights or results that could be the seed for future research or publication.
  \item Justify your choice of the new problem and explain how the method applies to it.
  \item If modifications were necessary, describe them and explain why.
\end{itemize}

\section*{Report Format and Submission Guidelines}

\begin{enumerate}
  \item \textbf{Title Page}
  \begin{itemize}
    \item Project title
    \item Team members (names and student IDs)
    \item Assigned paper citation
    \item Date of submission
  \end{itemize}

  \item \textbf{Abstract} (150--200 words)
  \begin{itemize}
    \item A concise summary of the report, including objectives, methodology, and key outcomes.
  \end{itemize}

  \item \textbf{Introduction}
  \begin{itemize}
    \item Background on the optimization method
    \item Summary of the assigned paper
    \item Objectives of the project
  \end{itemize}

  \item \textbf{Reproducible Research (Part 1)}
  \begin{itemize}
    \item Description of the algorithm and implementation
    \item Tools/libraries used
    \item Results and comparisons with original paper
    \item Observations and analysis
  \end{itemize}

  \item \textbf{Extension via Code Recycling (Part 2)}
  \begin{itemize}
    \item Description of the new problem
    \item Justification for using the original method (or a variant)
    \item Adjustments to the original algorithm (if any)
    \item Implementation details, results, and insights
    \item Critical discussion
  \end{itemize}

  \item \textbf{Conclusion}
  \begin{itemize}
    \item Summary of what was learned
    \item Reflections on both parts of the project
    \item Potential for future work
  \end{itemize}

  \item \textbf{References}
  \begin{itemize}
    \item Full citations of all sources used
  \end{itemize}

  \item \textbf{Appendix (Optional)}
  \begin{itemize}
    \item Source code (or a link to a GitHub repository)
    \item Additional figures or tables
    \item Data descriptions
  \end{itemize}
\end{enumerate}

\vspace{1cm}
\hrule

\bigskip
\bigskip

\begin{center}
{\it Please use the template below to format your final report.}
\end{center}

\section*{Abstract}
A brief summary of the objectives, methods, and main findings of the project. (150--200 words)

\section{Introduction}
Provide background on the optimization method, summarize the assigned paper, and outline the objectives of your project.

\section{Reproducible Research (Part 1)}
\subsection{Method Overview}
Describe the optimization algorithm used in the assigned paper.

\subsection{Implementation Details}
Detail your implementation approach, tools, and any assumptions.

\subsection{Results and Comparison}
Present and compare your results with those in the original paper. Use figures/tables as needed.

\subsection{Analysis and Discussion}
Discuss discrepancies, difficulties, and insights. Comment on the method's strengths and weaknesses.

\section{Code Recycling and Extension (Part 2)}
\subsection{New Problem Description}
Describe the new engineering or applied science problem.

\subsection{Method Justification and Adaptation}
Explain why and how the original method applies or was modified.

\subsection{Implementation and Results}
Provide implementation details, results, and analysis.

\subsection{Critical Discussion}
Reflect on what was learned, the effectiveness of the approach, and possible future improvements.

\section{Conclusion}
Summarize your findings and learning outcomes from both parts of the project.

\section*{References}
Include full citations for all papers, datasets, and tools used.

\section*{Appendix (Optional)}
Include any code snippets, extra figures, or data explanations. You may also provide a GitHub link to your project repository.

\end{document}